\usepackage[left=2cm, right=2cm, top=3cm, bottom=3cm]{geometry} %Para poner los márgenes


\usepackage{float} %Para hacer que la megatabla no se vaya volando a la última página y forzar que se quede usando [H]

% Idioma español y codificación
\usepackage[spanish]{babel} % Traduce palabras como "Contenido", "Capítulo", etc.
\usepackage[utf8]{inputenc} % Permite escribir caracteres en español (á, é, ñ, etc.)
\usepackage[T1]{fontenc}    % Mejora la salida de caracteres acentuados en PDF

% Paquetes matemáticos
\usepackage{amsmath}  % Permite usar comandos avanzados de matemáticas
\usepackage{amssymb}  % Agrega símbolos matemáticos adicionales
\usepackage{amsthm}   % Permite definir teoremas, lemas, etc.
\usepackage{mathrsfs} % Fuente adicional para letras matemáticas

% Para mejor tipografía en matemáticas
\usepackage{mathtools}

% Si necesitas gráficos
\usepackage{graphicx} 

% Si usas referencias cruzadas en el documento
\usepackage[hidelinks]{hyperref}  % Enlaces clicables en PDF

%Para poner el color de los enlaces azul
\hypersetup{
    colorlinks=true,       % Habilita los enlaces de colores
    linkcolor=black,        % Color de los enlaces internos (por ejemplo, referencias a secciones)
    urlcolor=magenta,      % Color de los enlaces a direcciones web
    citecolor=red,         % Color de las referencias bibliográficas
    filecolor=cyan,        % Color de los enlaces a archivos
}

\renewcommand{\contentsname}{Índice} % Cambia "Contenido" por "Índice"

\usepackage{listings}
\usepackage{xcolor} % Para definir colores, opcional

% Configuración básica de listings:
\lstset{
    language=C,                    % Especifica el lenguaje (puede ser Python, Java, etc.)
    basicstyle=\ttfamily\small,     % Fuente de tipo monoespaciada
    numbers=left,                  % Números de línea a la izquierda
    numberstyle=\tiny,             % Estilo de los números de línea
    stepnumber=1,                  % Mostrar número en cada línea
    numbersep=10pt,                % Espacio entre número y código
    backgroundcolor=\color{lightgray}, % Color de fondo
    showspaces=false,              % No resaltar espacios
    showstringspaces=false,        % No resaltar espacios en cadenas
    showtabs=false,                % No resaltar tabulaciones
    frame=single,                  % Agregar un marco alrededor del código
    rulecolor=\color{black},       % Color del marco
    captionpos=b,                  % Posición de la leyenda (b de bottom)
    breaklines=true,               % Saltos automáticos de línea
    breakatwhitespace=false,
    title=\lstname,                % Muestra el nombre del archivo, si se especifica
    keywordstyle=\color{blue},     % Color de las palabras clave
    commentstyle=\color{green},    % Color de los comentarios
    stringstyle=\color{red}        % Color de las cadenas
}

\usepackage{caption}
\captionsetup[table]{labelformat=empty} %Para que no salgan numeradas las tablas
\captionsetup[lstlisting]{labelformat=empty} %Para que no salgan numeradas los fragmentos de código
\captionsetup[table]{labelformat=empty} %Para que no salgan numeradas las tablas
\captionsetup[figure]{labelformat=empty} %Para que no salgan numeradas las gráficas
\setlength{\parindent}{0.75cm} %Para que la sangría no sea tan grande
