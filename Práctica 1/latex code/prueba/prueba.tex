\documentclass[a4paper,12pt]{article} % Puedes cambiar 'article' por otro tipo de documento

% Idioma español y codificación
\usepackage[spanish]{babel} % Traduce palabras como "Contenido", "Capítulo", etc.
\usepackage[utf8]{inputenc} % Permite escribir caracteres en español (á, é, ñ, etc.)
\usepackage[T1]{fontenc}    % Mejora la salida de caracteres acentuados en PDF

% Paquetes matemáticos
\usepackage{amsmath}  % Permite usar comandos avanzados de matemáticas
\usepackage{amssymb}  % Agrega símbolos matemáticos adicionales
\usepackage{amsthm}   % Permite definir teoremas, lemas, etc.
\usepackage{mathrsfs} % Fuente adicional para letras matemáticas

% Para mejor tipografía en matemáticas
\usepackage{mathtools}

% Si necesitas gráficos
\usepackage{graphicx} 

% Si usas referencias cruzadas en el documento
\usepackage[hidelinks]{hyperref}  % Enlaces clicables en PDF

%Para poner el color de los enlaces azul
\hypersetup{
    colorlinks=true,       % Habilita los enlaces de colores
    linkcolor=black,        % Color de los enlaces internos (por ejemplo, referencias a secciones)
    urlcolor=magenta,      % Color de los enlaces a direcciones web
    citecolor=red,         % Color de las referencias bibliográficas
    filecolor=cyan,        % Color de los enlaces a archivos
}

\renewcommand{\contentsname}{Índice} % Cambia "Contenido" por "Índice"

\usepackage{listings}
\usepackage{xcolor} % Para definir colores, opcional

% Configuración básica de listings:
\lstset{
    language=C,                    % Especifica el lenguaje (puede ser Python, Java, etc.)
    basicstyle=\ttfamily\small,     % Fuente de tipo monoespaciada
    numbers=left,                  % Números de línea a la izquierda
    numberstyle=\tiny,             % Estilo de los números de línea
    stepnumber=1,                  % Mostrar número en cada línea
    numbersep=10pt,                % Espacio entre número y código
    backgroundcolor=\color{lightgray}, % Color de fondo
    showspaces=false,              % No resaltar espacios
    showstringspaces=false,        % No resaltar espacios en cadenas
    showtabs=false,                % No resaltar tabulaciones
    frame=single,                  % Agregar un marco alrededor del código
    rulecolor=\color{black},       % Color del marco
    captionpos=b,                  % Posición de la leyenda (b de bottom)
    breaklines=true,               % Saltos automáticos de línea
    breakatwhitespace=false,
    title=\lstname,                % Muestra el nombre del archivo, si se especifica
    keywordstyle=\color{blue},     % Color de las palabras clave
    commentstyle=\color{green},    % Color de los comentarios
    stringstyle=\color{red}        % Color de las cadenas
}


\begin{document}
\begin{titlepage}
	\centering

	\includegraphics[width=0.3\textwidth]{logo_ugr.png} % Ajusta el tamaño según necesites (0.3\textwidth es solo un ejemplo)
	\vspace{1cm} % Ajusta el espacio según lo necesites

	{\bfseries\LARGE Universidad de Granada \par}
	\vspace{1cm}
	{\scshape\Large Facultad de Ingeniería Informática y telecomunicaciones \par}
	\vspace{3cm}
	{\scshape\Huge Práctica 1: Eficiencia de algoritmos \par}
	\vspace{3cm}
	{\itshape\Large Doble Grado Ingeniería Informática y Matemáticas \par}
	\vfill
	{\Large Autores: \par}
	{\Large Adolfo Martínez Olmedo, Pablo Delgado Galera, Marcos Baena Solar \par}
	\vfill
	{\Large Marzo 2025 \par}
\end{titlepage}

\tableofcontents
\newpage


% Sections
\section{Introducción}
Comenzemos estableciendo las características de cada uno de nuetros
ordenadores, ya que tienen prestaciones diferentes

\begin{table}[h!]
	\centering
	\makebox[\textwidth]{%
		\scalebox{0.6}{%
			\begin{tabular}{|c|c|c|c|c|c|c|c|}
				\hline
				\textbf{}       & \textbf{CPU}                         & \textbf{RAM} & \textbf{Caché L1d}    & \textbf{Caché L1i}    & \textbf{Caché L2}    & \textbf{Caché L3}   & \textbf{SO}        \\ \hline
				Adolfo Martínez & AMD Ryzen 7 4800HS                   & 16GB         & 256 KiB               & 256 KiB               & 4 MiB                & 8 MiB               & Ubuntu 22.04.4     \\ \hline
				Marcos Baena    & AMD 3020e with Radeon Graphics       & 5,7Gi        & 64 KiB (2 instances)  & 128 KiB (2 instances) & 1 MiB (2 instances)  & 4 MiB (1 instance)  & Ubuntu 24.04.1 LTS \\ \hline
				Pablo Delgado   & 11th Gen Intel(R) Core(TM) i7-11800H & 15 Gi        & 384 KiB (8 instances) & 256 KiB (8 instances) & 10 MiB (8 instances) & 24 MiB (1 instance) & Ubuntu 24.04.1     \\ \hline
			\end{tabular}
		}
	}
	\caption{Características de los ordenadores}
	\label{tab:caracteristicas}
\end{table}

En esta práctica vamos a discutir la eficiencia de los algoritmos
desde tres puntos de vista distintos:

\begin{itemize}
	\item \textbf{Punto 1:} Descripción del primer punto.
	\item \textbf{Punto 2:} Descripción del segundo punto.
	\item \textbf{Punto 3:} Descripción del tercer punto.
\end{itemize}

\subsection{Análisis de la eficiencia teórica}
En el análisis de la eficiencia teórica estudiaremos el tiempo de ejecución del algoritmo
mediante funciones en notación \textit{Big-O}, que representarán el peor caso posible.
En este análisis, no usaremos medidas reales de computación, sino que calcularemos funciones
mediante técnicas vistas en Estructuras de Datos y Algorítmica.
\subsection{Análisis de la eficiencia empírica}
Para el análisis de la eficiencia empírica ejecutaremos los algoritmos implementados en C++ en cada
una de nuestras máquinas y mediermos el tiempo de ejecución mediante la
clase <chrono>. Cada miembro del equipo ejecutará cada algoritmo 10 veces con todos los tamaños especificados,
para luego hacer una media y obtener resultados más fiables.
\subsection{Análisis de la eficiencia híbrida}
En el análisis de la eficiencia hibrída, tomammos los resultados de los integrantes del grupo y hallamos
la constante $\kappa$. En la representación de los resultados usaremos la herramienta gnuplot.

Para poder completar esta parte del estudio de la eficiencia usaremos los resultados del análisis teórico, para
poder conocer la forma de la función a la que queremos ajustar los datos. Por ejemplo para representar en gnuplot $O(n^{3})$:

\begin{lstlisting}[language=C, caption={Ejemplo de $O(n^{2})$}]
       gnuplot> f(x) = a0*x*x+a1*x*x+a2
        \end{lstlisting}

Después de esto debemos hacer la regresión usando el método de mínimos, cuyo funcionamiento conoces gracias a la asignatura EDIP:

\begin{lstlisting}[language=C, caption={Uso de gnuplot para l regresión}]
        gnuplot> fit f(x) 'result.dat' via a0,a1,a2
         \end{lstlisting}

En este caso result.dat es nuestro fichero de datos.
Nos centraremos en la parte que pone \textit{Final set of Parameters}, que nos muestra los coeficientes de la fórmula de regresión
junto con la bondad del ajuste realizado.

Finalmente, hacemos el plot de los puntos y la curva de ajuste para ver como de buena es el cálculo de la eficiencia híbrida. Usaremos el
siguiente comando:

\begin{lstlisting}[language=C, caption={Representación de la regresión}]
        gnuplot> plot 'result.dat', f(x) title 'Curva de ajuste'
         \end{lstlisting}




\section{Desarrollo}
Una vez que hemos discutido las maneras de estudiar la eficiencia, veamos los problemas
que vamos a analizar: La \textbf{Ordenación de vectores},
los \textbf{Números de Catalan}, y las \textbf{Torres de Hanoi.}

\subsection{Ordenación de vectores}



\subsection{Los números de Catalan}

\subsubsection{Versión recursiva}

\subsubsection{Versión iterativa (programación dinámica)}

\subsubsection{Versión iterativa directa usado el coeficiente binomial}

\subsection{Las Torres de Hanoi}

\subsubsection{Versión recursiva}

\subsubsection{Versión iterativa usando una pila}

\subsubsection{Versión iterativa sin usar la pila}

\section{Conclusión}
En conclusión, somos sudorosos.




\end{document}