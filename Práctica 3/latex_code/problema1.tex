\subsection{Elementos}

Para este problema los elementos que pertenecen al mismo son los siguientes:

\begin{itemize}[label=\textcolor{red}{\textbullet}]
	\item   \textcolor{red}{\textbf{Conjuntos de candidatos:}} los \textit{n} contenedores.
	\item \textcolor{red}{\textbf{Conjunto de candidatos elegidos:}} la solución es un vector $(x_1, x_2,\cdots , x_n)$ que indica los contenedores que se meten en el buque mercante.
	\item \textcolor{red}{\textbf{Función solución:}} Comprueba que el buque esté completamente lleno:
	      \begin{equation*}
		      \sum_{i = 1}^{n} x_ip_i = \mathcal{P}
	      \end{equation*}
	\item \textcolor{red}{\textbf{Función de factibilidad:}} el peso de los contenderos metidos en el buque hasta el momento más el peso del objeto más prometedor que no supero el peso total del buque.
	\item \textcolor{red}{\textbf{Función objetivo:}} beneficio total de los objetos introducidos en el buque:
	      \begin{equation*}
		      \sum_{i = 1}^{n} x_ib_i
	      \end{equation*}
\end{itemize}

\subsection{Diseño del algoritmo}

\subsection{Estudio de optimalidad}

\subsection{Ejemplo}
