\documentclass[a4paper,12pt]{article}

\usepackage[left=2cm, right=2cm, top=3cm, bottom=3cm]{geometry} %Para poner los márgenes


\usepackage{float} %Para hacer que la megatabla no se vaya volando a la última página y forzar que se quede usando [H]

% Idioma español y codificación
\usepackage[spanish]{babel} % Traduce palabras como "Contenido", "Capítulo", etc.
\usepackage[utf8]{inputenc} % Permite escribir caracteres en español (á, é, ñ, etc.)
\usepackage[T1]{fontenc}    % Mejora la salida de caracteres acentuados en PDF

% Paquetes matemáticos
\usepackage{amsmath}  % Permite usar comandos avanzados de matemáticas
\usepackage{amssymb}  % Agrega símbolos matemáticos adicionales
\usepackage{amsthm}   % Permite definir teoremas, lemas, etc.
\usepackage{mathrsfs} % Fuente adicional para letras matemáticas

% Para mejor tipografía en matemáticas
\usepackage{mathtools}

% Si necesitas gráficos
\usepackage{graphicx} 

% Si usas referencias cruzadas en el documento
\usepackage[hidelinks]{hyperref}  % Enlaces clicables en PDF

%Para poner el color de los enlaces azul
\hypersetup{
    colorlinks=true,       % Habilita los enlaces de colores
    linkcolor=black,        % Color de los enlaces internos (por ejemplo, referencias a secciones)
    urlcolor=magenta,      % Color de los enlaces a direcciones web
    citecolor=red,         % Color de las referencias bibliográficas
    filecolor=cyan,        % Color de los enlaces a archivos
}

\renewcommand{\contentsname}{Índice} % Cambia "Contenido" por "Índice"

\usepackage{listings}
\usepackage{xcolor} % Para definir colores, opcional

% Configuración básica de listings:
\lstset{
    language=C,                    % Especifica el lenguaje (puede ser Python, Java, etc.)
    basicstyle=\ttfamily\small,     % Fuente de tipo monoespaciada
    numbers=left,                  % Números de línea a la izquierda
    numberstyle=\tiny,             % Estilo de los números de línea
    stepnumber=1,                  % Mostrar número en cada línea
    numbersep=10pt,                % Espacio entre número y código
    backgroundcolor=\color{lightgray}, % Color de fondo
    showspaces=false,              % No resaltar espacios
    showstringspaces=false,        % No resaltar espacios en cadenas
    showtabs=false,                % No resaltar tabulaciones
    frame=single,                  % Agregar un marco alrededor del código
    rulecolor=\color{black},       % Color del marco
    captionpos=b,                  % Posición de la leyenda (b de bottom)
    breaklines=true,               % Saltos automáticos de línea
    breakatwhitespace=false,
    title=\lstname,                % Muestra el nombre del archivo, si se especifica
    keywordstyle=\color{blue},     % Color de las palabras clave
    commentstyle=\color{green},    % Color de los comentarios
    stringstyle=\color{red}        % Color de las cadenas
}

\usepackage{caption}
\captionsetup[table]{labelformat=empty} %Para que no salgan numeradas las tablas
\captionsetup[lstlisting]{labelformat=empty} %Para que no salgan numeradas los fragmentos de código
\captionsetup[table]{labelformat=empty} %Para que no salgan numeradas las tablas
\captionsetup[figure]{labelformat=empty} %Para que no salgan numeradas las gráficas
\setlength{\parindent}{0,75cm} %Para que la sangría no sea tan grande

\usepackage[utf8]{inputenc}
\usepackage{amsmath, amsfonts, amssymb}  % Paquetes de matemáticas
\usepackage{graphicx}                    % Para insertar figuras
\usepackage{hyperref}                    % Para enlaces y referencias clicables
\usepackage{geometry}                    % Para márgenes y diseño de página
\usepackage{float}                       % Para forzar la posición de las figuras

\geometry{
    left=2.5cm,
    right=2.5cm,
    top=3cm,
    bottom=3cm
}


% =====================
% Índice
% =====================
\begin{document}
\begin{titlepage}
	\centering

	\includegraphics[width=0.3\textwidth]{./Imagenes/logo_ugr.png} % Ajusta el tamaño según necesites (0.3\textwidth es solo un ejemplo)
	\vspace{1cm} % Ajusta el espacio según lo necesites

	{\bfseries\LARGE Universidad de Granada \par}
	\vspace{1cm}
	{\scshape\Large Facultad de Ingeniería Informática y telecomunicaciones \par}
	\vspace{3cm}
	{\scshape\Huge Práctica 2: Divide y Vencerás \par}
	\vspace{3cm}
	{\itshape\Large Doble Grado Ingeniería Informática y Matemáticas \par}
	\vfill
	{\Large Autores: \par}
	{\Large Adolfo Martínez Olmedo, Pablo Delgado Galera, Marcos Baena Solar \par}
	\vfill
	{\Large Marzo 2025 \par}
\end{titlepage}

\tableofcontents
\newpage


% =====================
\section{Introducción}
% =====================
En esta sección, se incluye una breve descripción del tema de estudio: 
por qué son importantes los algoritmos Divide y Vencerás, 
dónde se aplican y qué se espera conseguir con la práctica.

% =====================
\section{Objetivos}
% =====================
Detallar brevemente cuáles son los objetivos que se persiguen con la realización de la práctica:
\begin{itemize}
    \item Comprender la técnica de Divide y Vencerás y sus ventajas.
    \item Comparar con la estrategia de fuerza bruta y analizar la complejidad.
    \item Implementar ambos enfoques (fuerza bruta y Divide y Vencerás) para cada problema.
    \item Experimentar con el umbral de la técnica de Divide y Vencerás.
\end{itemize}

% =====================
\section{Problemas a Resolver}
% =====================
En esta sección se describen los diferentes problemas que se abordarán.

\subsection{El número más pequeño}
Describir el problema de obtener el entero más pequeño de $k$ cifras 
a partir de un vector de dígitos. Mencionar requisitos, entrada y salida.

\subsection{El par de puntos más cercano}
Describir el problema de encontrar dos puntos con la mínima distancia 
Euclídea dentro de un conjunto. Explicar la versión de fuerza bruta 
y la necesidad de mejorar su eficiencia.

\subsection{La envolvente convexa}
Explicar en qué consiste la envolvente convexa de un conjunto de puntos, 
así como la forma de resolverla mediante algoritmos de fuerza bruta y 
Divide y Vencerás.

% =====================
\section{Metodología y Diseño de los Algoritmos}
% =====================
Explicar detalladamente cómo abordar cada problema tanto con fuerza bruta 
como con Divide y Vencerás.

\subsection{Análisis de Fuerza Bruta}
\begin{itemize}
    \item Descripción general de la estrategia de fuerza bruta.
    \item Complejidad temporal para cada uno de los tres problemas.
    \item Ventajas y desventajas.
\end{itemize}

\subsection{Análisis de Divide y Vencerás}
\begin{itemize}
    \item Descripción de la plantilla general de Divide y Vencerás.
    \item Aplicación concreta para cada problema (pasos de división, recursión y combinación).
    \item Complejidad temporal teórica y justificación.
    \item Discusión sobre el umbral para comparar cuándo es mejor la aproximación recursiva.
\end{itemize}

\subsection{Detalles de Implementación}
\begin{itemize}
    \item Lenguaje de programación utilizado (por ejemplo, C++).
    \item Estructura de los ficheros (módulos, cabeceras, etc.).
    \item Consideraciones sobre la lectura de datos, tratamiento de casos límite, etc.
\end{itemize}

% =====================
\section{Resultados Experimentales}
% =====================
En esta sección se incluyen los experimentos realizados para comparar 
las implementaciones de fuerza bruta y Divide y Vencerás.

\subsection{Configuración de las Pruebas}
\begin{itemize}
    \item Descripción del entorno de ejecución (CPU, memoria, compilador, etc.).
    \item Conjunto de datos utilizados para las pruebas (tamaño, forma de generarlos).
\end{itemize}

\subsection{Tablas y Gráficos de Rendimiento}
Insertar aquí las tablas y/o gráficos que muestren los tiempos de ejecución, 
uso de memoria, etc.

\subsection{Análisis de los Resultados}
\begin{itemize}
    \item Comparación cualitativa (efectividad, facilidad de implementación).
    \item Comparación cuantitativa (tiempos de ejecución, consumo de memoria).
    \item Conclusiones sobre el umbral experimental.
\end{itemize}

% =====================
\section{Conclusiones}
% =====================
Resumen de los hallazgos principales. Mencionar qué se aprendió 
y qué aspectos se pueden mejorar o extender.

% Me imagino que esto no va a hacer falta
% =====================
\section{Bibliografía}
% =====================
Citar las referencias utilizadas para la elaboración de la práctica, 
por ejemplo, libros, apuntes de clase o fuentes externas:

\begin{thebibliography}{9}
    \bibitem{cormen}
    Thomas H. Cormen, Charles E. Leiserson, Ronald L. Rivest, and Clifford Stein, 
    \textit{Introduction to Algorithms}. MIT Press, 3rd Edition, 2009.
    
    \bibitem{otros}
    Apuntes de la asignatura de Algorítmica, Departamento de Ciencias de la Computación e I.A., 
    Universidad de Granada.
\end{thebibliography}

\end{document}
